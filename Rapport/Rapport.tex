\documentclass[12pt,a4paper]{scrartcl}
\usepackage[utf8]{inputenc}
\usepackage[T1]{fontenc}
\usepackage[french]{babel}
\usepackage{textcomp}
\usepackage{lmodern}
\usepackage{graphicx}
\usepackage[dvipsnames,svgnames]{xcolor}
\usepackage{microtype}
\usepackage{listings}
\usepackage{hyperref} \hypersetup{colorlinks=true,linkcolor=Brown,urlcolor=Navy,
breaklinks=true,pdfstartview=XYZ}
\usepackage{fancyhdr} 
\title{Rapport projet I62}
\author{Tom Bartier, Nino Cantera, Balbony Ivano}

\begin{document}
\pagestyle{fancy}
\fancyhead{} % clear all header fields
\fancyhead[RO,LE]{\textbf{Compte Rendu TP I62}}

\maketitle


\begin{abstract}
Ce rapport concerne le projet du module I62 Génie Logiciel de la L3 Informatique
de l'université de Toulon. Le but de ce projet était de nous faire découvrir
les différents concepts théoriques du génie logiciel et de les appliquer sur un projet 
pratique travaillé pendant les séances de TP ainsi qu'en dehors des heures de cours.
\end{abstract}

\tableofcontents

\section {Introduction}

Notre projet était de mettre au point un simulateur de rover explorant la planète Mars.
L'application est en mode client / serveur, elle permet à l'utilisateur de contrôler un rover
et d'explorer la planète Mars avec d'autres utilisateurs, d'analyser différents matériaux
ou encore de déployer un drone afin de faciliter l'exploration.

\subsection{Infos Pratiques}
L'application ne peut se lancer que si le serveur est allumé. 
Pour lancer le serveur, executez la commande \lstinline!python ./src/server/app-server.py <ip> <port>!
où <ip> correspond à l'adresse ip de votre machine et <port> désigne le port sur lequel le
serveur attendra les connexions. Exemple : \lstinline!python ./src/server/app-server.py 127.0.0.1 25025! \\\\
Pour lancer l'application client, executez la commande python ./src/client/app-client.py <ip> <port>

Côté client vous arrivez sur la page de connexion, deux comptes sont disponibles :
\begin{enumerate}
    \item Utilisateur : bob     Mot de Passe : 1234
    \item Utilisateur : alice   Mot de Passe : 5678
\end{enumerate}

Si la connexion ne fonctionne pas c'est que vous avez entré la mauvaise IP / Port, ou que le serveur n'était pas allumé lorsque vous avez lancé le client.\\
Vous devez arriver sur l'écran principal, utilisez les flèches directionnelles pour vous déplacer et cliquez sur les boutons à l'écran pour 
changer entre le rover et l'hélicoptère, ou pour casser et analyser un rocher. À noter que pour ranger l'hélicoptère vous devez vous placer sur le rover, et pour 
analyser un rocher vous devez être devant orienté dans la bonne direction. Il n'y a pas encore de manière propre de quitter le client et le serveur, fermez juste la fenêtre
du client et le terminal du serveur.

\section{Informations générales}
Ce projet est donc un simulateur de rover sur Mars écrit en Python et qui a été séparé en deux grandes phases : 
une première phase de conception en séance de TP où l'on a rélféchi à la conception du projet en réalisant
les différents diagrammes UML vu en cours (voir \ref{diagrammes}). C'est durant cette phase que nous nous sommes mis d'accord
sur la structuration du projet ainsi que les différentes fonctionnalités proposées.\\
Le logiciel est dans l'idée destiné à une entreprise pour un usage privé, il n'est donc pas disponible
au public, c'est pourquoi nous avons décidé de ne pas implémenter de fonction pour créer un compte directement depuis
l'écran de connexion. L'application administrateur n'a pas pu être développée par manque de temps, mais elle est prévue
dans la conception, elle est très fortement liée au serveur, c'est à dire qu'elle se lance sur la même machine que le serveur
en même temps et elle est unique, il n'y a donc qu'un seul administrateur. C'est lui qui doit créer un nouveau compte si une nouvelle
personne veut avoir accès au simulateur.\\
Parmi les différentes fonctionnalités imaginées mais que nous n'avons pas pu mettre en place, nous pouvons citer
la prise de photos et vidéo, que nous n'avons pas jugés pertinents d'implémenter dans notre programme en 2D, mais si 
le simulateur avait été fait en 3D cette fonctionnalité aurait eu sa place. Nous avons aussi imaginé un système d'exploration
en équipe avec d'autres utilisateurs : chaque utilisateur aurait une représentation minimaliste de la planète dans un coin de l'écran
donnant des informations sur l'emplacement actuel de chaque utilisateur ainsi que les zones qu'ils ont explorées. Les zones
pas encore découvertes seraient cachées et lorsque qu'un utilisateur découvrirait une nouvelle zone les informations sur cette dernière
seraient retransmises à tous les autres utilisateurs.\\
Nous avons aussi imaginé et essayé d'implanter un système de topographie. La carte de Mars que nous avons est colorée de façon
à représenter la topographie de Mars (voir \ref{sources}). Nous avons essayé de déterminer un niveau de hauteur en fonction de la couleur de chaque pixel
afin de simuler la présence de montagnes et cratères. le rover serait gêné dans sa progression par ces derniers, d'où l'interêt de déployer
le drone afin de faciliter l'exploration sur des terrains escarpés. Mais l'implémentation de ce système a posé trop de problèmes, il est cependant 
toujours présent dans le code source mais non utilisé.\\
Nous avons aussi mis en place un système permettant de simuler la météo sur Mars. (DETAILLER LA METEO ICI)
\\
Concernant la mise en place du serveur, nous nous sommes basés sur un modèle de serveur déjà existant (voir \ref{sources})
que nous avons modifiés afin de répondre à nos attentes. Il utilise la librairie socket de Python ainsi que selectors, qui est 
une manière de faire un serveur asynchrone non bloquant en évitant d'avoir à gérer plusieurs threads.
Nous utilisons une unique classe Message transitant sur le réseau et contenant les diférentes requêtes et réponses, mais nous
ne l'avons pas mentionné dans les diagrammes car nous le considérons comme une API, un peu commme Tkinter, nous avons donc pas jugés
nécessaire de faire apparaitre cette classe dans les diagrammes.

Nous avons aussi pensés au fait que le serveur communiquerait avec une base de données. Toutes les données concernant les utilisateurs seraient stockées dans une base de donnée SQL
mais par manque de temps nous avons dans notre programme simulés cette base de données avec un fichier json dans lequel nous lisons les données à l'allumage du serveur et écrivons les nouvelles
données à stocker (en théorie) périodiquement ainsi qu'à l'extinction du serveur.\\
Concernant la répartition des tâches, Nino Cantera s'est occupé d'une grande partie de l'IHM, Ivano Balbony s'est occupé de la génération de la topographie, de la météo et de quelques 
classes serveur et Tom BARTIER s'est chargé de la communication entre le client et le serveur, le traitement des requêtes des deux côtés, quelques classes serveur et quelques éléments de l'IHM.
La quasi totalité du code a été écrit en dehors des séances de TP.


\section{Exigences}
Voici les exigences mises en place pour ce projet, à noter qu'elles n'ont pas toutes pu être implémentées
mais qu'elles pourraient l'être si le projet venait à être poursuivi après ce rendu.

\subsection{Interactions client/serveur}

\begin{itemize}

\item EX\_SERV\_0002
\begin{itemize}
\item S'authentifier
\item Le SI doit permettre à l'utilisateur de s'authentifier grâce à un
		identifiant et un mot de passe.
\end{itemize}


\item EX\_SERV\_0003
\begin{itemize}
\item Voir les autres
\item Le SI doit permettre à l'utilisateur de voir les autres autres 
	utilisateur dans l'environement
\end{itemize}

\item EX\_SERV\_0004
\begin{itemize}
\item Stockage environement
\item Le SI héberge sur le serveur l'environement dans lequel évoluent les 
	rovers.
\end{itemize}


\end{itemize}

\subsection{GUI}

\begin{itemize}

\item EX\_GUI\_0001
\begin{itemize}
\item Avancer
\item Le SI doit permettre à l'utilisateur de faire avancer le rover.
\end{itemize}

\item EX\_GUI\_0002
\begin{itemize}
\item Reculer
\item Le SI doit permettre à l'utilisateur de faire reculer le rover.
\end{itemize}

\item EX\_GUI\_0003
\begin{itemize}
\item Pivoter
\item Le SI doit permettre à l'utilisateur de faire pivoter le rover dans les 			deux sens (sens horaire et antihoraire).
\end{itemize}

\item EX\_GUI\_0004
\begin{itemize}
\item Contrôler la vitesse
\item Le SI doit permettre à l'utilisateur de contrôler la vitesse du rover.
\end{itemize}

\item EX\_GUI\_005
\begin{itemize}
\item Utiliser Laser
\item Le SI doit permettre à l'utilisateur de tirer un laser sur un rocher.
\end{itemize}


\item EX\_GUI\_0006
\begin{itemize}
\item Surveiller énergie
\item Le SI doit permettre à l'utilisateur de surveiller la quantité d'énergie
		restante au rover.
\end{itemize}

\item EX\_GUI\_0007
\begin{itemize}
\item Analyser
\item Le SI doit permettre à l'utilisateur d'obtenir les informations sur la 
		matière pulvérisée et analysée par le rover.
\end{itemize}

\item EX\_GUI\_0008
\begin{itemize}
\item Creuser
\item Le SI doit permettre à l'utilisateur d'indiquer au rover de creuser dans
		le sol à l'aide de sa foreuse
\end{itemize}

\item EX\_GUI\_0009
\begin{itemize}
\item Pilote Automatique
\item Le SI doit permettre à l'utilisateur de mettre le rover en mode pilote 
		automatique.
\end{itemize}		

\item EX\_GUI\_00010
\begin{itemize}
\item Mode Manuel
\item Le SI doit permettre à l'utilisateur de mettre le rover en mode manuel.
\end{itemize}
		
\item EX\_GUI\_0011
\begin{itemize}
\item Allumer
\item Le SI doit permettre à l'utilisateur d'allumer le rover.
\end{itemize}

\item EX\_GUI\_0012
\begin{itemize}
\item Eteindre
\item Le SI doit permettre à l'utilisateur d'éteindre le rover.
\end{itemize}

\item EX\_GUI\_0013
\begin{itemize}
\item Mini carte
\item Le SI doit permettre à l'utilisateur de consulter une mini carte de la planète (en ne voyant clairement que les zones découvertes).
\end{itemize}

\item EX\_GUI\_0014
\begin{itemize}
\item Affichage température
\item Le SI doit permettre à l'utilisateur de consulter la température actuelle 
		de la zone.
\end{itemize}

\end{itemize}

\subsection{Caméra}

\begin{itemize}


\item EX\_CAM\_0001
\begin{itemize}
\item Affichage Caméra
\item Le SI doit permettre à l'utilisateur de voir le retour de la caméra en 
		temps réel.
\end{itemize}

\item EX\_CAM\_0002
\begin{itemize}
\item Pivoter Caméra
\item Le SI doit permettre à l'utilisateur de faire pivoter la caméra dans 
	tous les sens.
\end{itemize}

\item EX\_CAM\_0003
\begin{itemize}
\item Zoom
\item Le SI doit permettre à l'utilisateur de zoomer et dézoomer la caméra. 
\end{itemize}


\item EX\_CAM\_0004
\begin{itemize}
\item Prendre des photos
\item Le SI doit permettre à l'utilisateur de prendre des photos avec la caméra 			et de les enregistrer sur sa machine.
\end{itemize}

\item EX\_CAM\_0005
\begin{itemize}
\item Prendre des vidéos
\item Le SI doit permettre à l'utilisateur de prendre des vidéos.
\end{itemize}

\item EX\_CAM\_0006
\begin{itemize}
\item Pivoter
\item Le SI doit permettre à l'utilisateur de faire pivoter la caméra.
\end{itemize}

\item EX\_CAM\_0007
\begin{itemize}
\item Enregistrer Photos
\item Le SI doit permettre à l'utilisateur d'enregistrer les photos prises.
\end{itemize}

\item EX\_CAM\_0008
\begin{itemize}
\item Enregistrer Photos
\item Le SI doit permettre à l'utilisateur d'enregistrer les vidéos prises.
\end{itemize}

\end{itemize}

\subsection{Rover}

\begin{itemize}

\item EX\_ROVER\_0001
\begin{itemize}
\item S'abîmer
\item Le SI doit infliger des dégâts au rover en cas de colision avec un obstacle ou en cas de chute.
\end{itemize}

\item EX\_ROVER\_0002
\begin{itemize}
\item Panne d'énergie
\item Le SI doit permettre au rover de tomber en panne d'énergie.
\end{itemize}

\item EX\_ROVER\_0003
\begin{itemize}
\item Recharge d'énergie
\item Le SI doit permettre au rover de se recharger en énergie.
\end{itemize}

\item EX\_ROVER\_0004
\begin{itemize}
\item Remplacer foreuse
\item Le SI doit permettre au rover d'abandonner sa foreuse et la remplacer par 
		une autre si la première se retrouve coincée ou endommagée.
\end{itemize}

\item EX\_ROVER\_0005
\begin{itemize}
\item Découvrir les alentours
\item Le SI doit permettre au rover de découvrir les alentours.
\end{itemize}

\item EX\_ROVER\_0006
\begin{itemize}
\item Prévoir météo
\item Le SI doit permettre au rover de prévoir de prévoir le prochain évènement
		météorologique de la zone où il se trouve.
\end{itemize}

\end{itemize}

\subsection {Hélicoptère}

\begin{itemize}

\item EX\_HELI\_0001
\begin{itemize}
\item Déployer hélicopter
\item Le SI doit permettre au rover de déployer l'hélicoptère
\end{itemize}

\item EX\_HELI\_0002
\begin{itemize}
\item Ranger hélicopter
\item Le SI doit permettre au rover de ranger l'hélicoptère
\end{itemize}

\item EX\_HELI\_0003
\begin{itemize}
\item Monter
\item Le SI doit permettre à l'utilisateur de faire monter l'hélicoptère
\end{itemize}

\item EX\_HELI\_0004
\begin{itemize}
\item Descendre
\item Le SI doit permettre à l'utilisateur de faire descendre l'hélicoptère
\end{itemize}

\item EX\_HELI\_0005
\begin{itemize}
\item Avancer
\item Le SI doit permettre à l'utilisateur de faire avancer l'hélicoptère
\end{itemize}

\item EX\_HELI\_0006
\begin{itemize}
\item Reculer
\item Le SI doit permettre à l'utilisateur de faire reculer l'hélicoptère
\end{itemize}

\item EX\_HELI\_0007
\begin{itemize}
\item Aller à gauche
\item Le SI doit permettre à l'utilisateur de faire aller l'hélicoptère vers la gauche
\end{itemize}

\item EX\_HELI\_0008
\begin{itemize}
\item Aller à droite
\item Le SI doit permettre à l'utilisateur de faire aller l'hélicoptère vers la droite
\end{itemize}

\item EX\_HELI\_0009
\begin{itemize}
\item Pivoter
\item Le SI doit permettre à l'utilisateur de faire pivoter l'hélicoptère
\end{itemize}

\item EX\_HELI\_0010
\begin{itemize}
\item Caméra Hélico
\item Le SI doit permettre à l'utilisateur contrôler la caméra de l'hélicoptère (voir rubrique Caméra)
\end{itemize}

\item EX\_HELI\_0011
\begin{itemize}
\item Energie
\item Le SI doit permettre à l'hécoptère de pouvoir tomber en panne d'énergie
\end{itemize}

\item EX\_HELI\_0012
\begin{itemize}
\item Energie
\item L'hécoptère doit pouvoir recharger son énergie en se posant rentrant à terre.
\end{itemize}

\item EX\_HELI\_0013
\begin{itemize}
\item Décollage
\item Le SI doit permettre à l'hélicoptère de décoller
\end{itemize}

\item EX\_HELI\_0014
\begin{itemize}
\item Aterrissage
\item Le SI doit permettre à l'hélicoptère d'atterrir
\end{itemize}

\item EX\_HELI\_0015
\begin{itemize}
\item Vitesse
\item Le SI doit permettre à l'utilisateur de changer la vitesse de l'hélicoptère
\end{itemize}

\item EX\_HELI\_0016
\begin{itemize}
\item Décollage
\item Le SI doit permettre à l'utilisateur de mettre l'hélicoptère en mode automatique.
\end{itemize}

\item EX\_HELI\_0016
\begin{itemize}
\item Cartographier
\item Le SI doit permettre à l'hélicoptère de cartographier ses environs
\end{itemize}

\end{itemize}

\subsection{Environnement}

\begin{itemize}

\item EX\_ENV\_0001
\begin{itemize}
\item Brouillard
\item Le SI affiche à l'utilisateur un brouillard dans les zones non découvertes par le rover et l'hélicoptère
\end{itemize}

\item EX\_ENV\_0002
\begin{itemize}
\item Rocher
\item Le SI dispose des rochers dans l'environement
\end{itemize}

\item EX\_ENV\_0002
\begin{itemize}
\item Rocher
\item Le SI dispose de différents niveaux de hauteur dans l'environement
\end{itemize}

\item EX\_ENV\_0003
\begin{itemize}
\item Tempête de poussière
\item Le SI dispose d'un évènement "tempête de poussière" sur une zone qui
		réduit la distance de vision et endommage un petit peu l'hélicoptère 
		lorsqu'il est en vol et empêche la recharge d'énergie solaire.
\end{itemize}

\item EX\_ENV\_0004
\begin{itemize}
\item Vent
\item Le SI dispose d'un évènement "vent" sur une zone avec différents niveaux d'intensité qui vont avoir une influence sur l'hélicoptère.
\end{itemize}

\end{itemize}

\section {Scénarios}

\subsection{Admin Modifier Environnement}
Scénario nominal :\\
1. L'administrateur clique sur le bouton "Modifier Environement"\\
2. Le système affiche l'interface de modification de l'environement\\
3. L'administrateur clique sur l'icône du rocher\\
4. Le système demande les coordonnées et le type du rocher à ajouter\\
5. L'administrateur renseigne les coordonnées auxquelles le rocher sera ajouté ainsi que le type\\
6. Le système confirme l'ajout du rocher sur la carte\\\\

Scénario alternatif : \\
3. L'administrateur clique sur l'icône tempête de poussière\\
4. Le système demande dans quelle zone sera la tempête de poussière ainsi que sa date et sa durée\\
5. L'administrateur renseigne les informations\\
6. Le système confirme l'ajout de la tempête de poussière\\\\

Scénario exception :\\
5. L'administrateur renseigne les coordonnées auxquelles le rocher sera ajouté ainsi que le type\\
6. Le système indique qu'un rocher est déjà présent à ces coordonnées, echec de l'ajout\\

\subsection{Scénario Client Piloter Hélico}
Scénario nominal:\\
\\
	1. L’utilisateur appuie sur la touche permettant de déployer l’hélicoptère\\
	2. L’hélicoptère virtuel se détache du rover et remplace temporairement le rover en tant que véhicule piloté par l’utilisateur\\
	3. Le système affiche la jauge d’énergie de l’hélicoptère\\
	4. L’utilisateur appuie sur la touche permettant d’avancer\\
	5. L’hélicoptère avance tant que la touche est enfoncée\\
	6. L’utilisateur relâche la touche avant que l’hélicoptère rencontre un obstacle\\
	7. L’hélicoptère arrête d’avancer\\
	8. L’utilisateur appuie sur la touche permettant de monter\\
	9. L’altitude de l’hélicoptère augmente\\
	10. L’utilisateur rappuie sur la touche pour avancer\\
	11. L’hélicoptère avance et passe au-dessus de l’obstacle\\
	12. L’utilisateur remarque que la jauge d’énergie de l’hélicoptère est basse\\
	13. L’utilisateur relâche la touche\\
	14. L’hélicoptère arrête d’avancer\\
	15. L’utilisateur appuie sur la touche permettant d’atterir\\
	16. L’hélicoptère atterrit et commence à recharger sa batterie\\
	17. L’utilisateur attend que la jauge d’énergie se remplisse\\
	18. Une fois la jauge remplie l’utilisateur appuie sur la touche permettant de décoller\\
	19. L’hélioptère se remet en route\\
	20. L’utilisateur appuie sur la touche permettant de pivoter\\
	21. L’hélicoptère pivote sur lui-même tant que la touche est enfoncée\\
	22. L’utilisateur relâche la touche\\
	23. L’hélicoptère arrête de pivoter\\
	25.  L’utilisateur appuie sur la touche permettant de ranger l’hélicoptère\\
	26. L’hélicoptère vient se poser sur le rover\\
	27. L’utilisateur reprend le contrôle du rover\\
\\
Cas alternatif: Contournement de l’obstacle\\
\\
	7. L’utilisateur appuie sur la touche permettant de pivoter\\
	8. L’hélicoptère pivote sur lui-même tant que la touche est enfoncée\\
	9. L’utilisateur relâche la touche après avoir fait pivoter l’hélicoptère à 45°\\
	10. L’utilisateur appuie sur la touche permettant d’avancer\\
	11. L’hélicoptère avance tant que la touche est enfoncée\\
	12. L’utilisateur relâche la touche après avoir fait parcourir une courte distance à l’hélicoptère\\
	13. L’hélicoptère arrête d’avancer\\
	12. L’utilisateur appuie sur la touche permettant de pivoter\\
	13. L’hélicoptère pivote sur lui-même tant que la touche est enfoncée\\
	14. L’utilisateur relâche la touche après avoir fait pivoter l’hélicoptère à 45°\\
	15. L’utilisateur appuie sur la touche permettant d’avancer\\
	16. L’hélicoptère avance tant que la touche est enfoncée\\
	17. L’utilisateur relâche la touche\\
	18. L’hélicoptère arrête d’avancer\\

	Le scénario reprend à l’étape 11. \\
\\
Cas d’exception: Batterie vide\\
\\
	11. L’utilisateur reste appuyé sur la touche pendant plusieurs minutes\\
	12. La jauge d’énergie de l’hélicoptère diminue et se vide\\
	13. L’hélicoptère tombe au sol\\
	14. L’hélicoptère perd des points de vie\\
	15. L’hélicoptère tombe à court de points de vie\\
	16. Le système indique que l’hélicoptère est trop endommagé pour décoller\\
	17. L’utilisateur reprend le contrôle du rover\\
	18. L’utilisateur appuie sur la touche permettant de contrôler l’hélicoptère\\
	19. Le système indique que l’hélicoptère est hors-service\\

\subsection{Scénario Piloter Rover}
Scénario nominal: Piloter le rover\\
\\
Scénario principal\\
\\
	1. L’utilisateur lance une partie en appuyant sur un bouton\\
	2. Le programme principal se lance et affiche l’interface\\
	3. L’utilisateur appuie sur la touche permettant d’avancer\\
	4. Le rover virtuel avance tant que la touche est enfoncée\\
	5. L’utilisateur relâche la touche\\
	6. Le rover s’arrête\\
	7. L’utilisateur appuie sur la touche permettant de reculer\\
	8. Le rover recule tant que la touche est enfoncée\\
	9. L’utilisateur relâche la touche\\
	10. Le rover s’arrête \\
	11. L’utilisateur appuie sur la touche permettant de pivoter\\
	12. Le rover pivote sur lui-même tant que la touche est enfoncée\\
	13. L’utilisateur relâche la touche\\
	14. Le rover s’arrête\\
	15. L’utilisateur appuie sur la touche permettant d’avancer\\
	16. Le rover avance tant que la touche est enfoncée\\
	17. L’utilisateur relâche la touche avant de percuter un rocher\\
	18. Le rover s’arrête\\
	19. L’utilisateur appuie sur la touche permettant de creuser\\
	20. Le rover utilise sa foreuse pour détruire le rocher et en collecter un échantillon\\
	21. L’utilisateur appuie sur la touche permettant de lancer une analyse\\
	22. Le résultat de l’analyse est affiché à l’écran\\
	23. L’utilisateur appuie sur la touche permettant de régler la vitesse\\
	24. L’utilisateur appuie sur la touche permettant d’augmenter la vitesse\\
	25. La jauge de vitesse augmente tant que la touche est enfoncée\\
	26. L’utilisateur relâche la touche\\
	27. La jauge de vitesse arrête d’augmenter\\
	28. L’utilisateur appuie sur la touche permettant de confirmer la vitesse choisie\\
	29. La vitesse choisie est appliquée\\
	30. L’utilisateur appuie sur la touche permettant d’avancer\\
	31. Le rover avance tant que la touche est enfoncée\\
	32. L’utilisateur relâche la touche\\
	33. Le rover s’arrête\\
	34. L’utilisateur clique sur le bouton permettant d’afficher le menu\\
	35. Le simulateur se met en pause et le menu s’affiche \\
	36. L’utilisateur clique sur le bouton «revenir à l’écran-titre»\\
	37. Le programme retourne à l’écran d’accueil\\
	38. L’utilisateur clique sur le bouton «déconnexion»\\
	39. L’utilisateur est déconnecté de son compte\\
	40. L’utilisateur ferme le programme\\
\\
Cas alternatif: Collision avec un obstacle\\
\\
	11. L’utilisateur reste appuyé sur la touche\\
	12. Le rover ne s’arrête pas et percute un rocher\\
	13. Le système indique que le rover est bloqué par un rocher\\
	14. Le rover est endommagé et perd des points de vie\\
\\
	Le scénario reprend à l’étape 19\\
\\
Cas d’exception: Echec de la mission\\
\\
	11. L’utilisateur reste appuyé sur la touche\\
	12. Le rover ne s’arrête pas et continue d’avancer jusqu’à tomber du haut d’une montagne\\
	13. Le rover tombe à court de points de vie\\
	14. Le système indique que le rover est trop endommagé pour continuer sa mission\\
	15. Le système indique que la partie est terminée\\
\\
	Le scénario reprend à l’étape 37\\




\section{Sources} \label{sources}
Base du modèle client / serveur (app-server.py, libserver.py, app-client.py, libclient.py) : \url{https://realpython.com/python-sockets/}\\
Rover: \url{https://www.researchgate.net/figure/D-Top-view-of-the-rover-with-dimensions_fig1_356749945}\\
Rochers: \url{https://www.pngwing.com/en/free-png-zrzud}\\
Hélicoptère: \url{https://www.creativefabrica.com/fr/png/drone-png-ORYOxgTGA/}\\
Mars (arrière-plan authentifiation): \url{https://www.frostscience.org/volcanic-activity-on-mars-upends-our-understanding-of-the-red-planets-interior/}\\
Carte: \url{https://attic.gsfc.nasa.gov/mola/images.html}\\
Tempête : \url{https://pngtree.com/freepng/strong-wind-vector_9038035.html}\\
Tempête de poussière : \url{https://www.vecteezy.com/png/21913770-brown-watercolor-modern-brush-style-with-colorful-texture-for-your-template}\\

\section{Diagrammes} \label{diagrammes}
Voici les différents diagrammes UML de la conception du projet.
\begin{figure}[b]
    \centering
    \includegraphics[width=0.5\textwidth]{DiagCS_Admin.png}
    \caption{Diagramme de Contexte Statique Serveur}\label{cs_serv}
\end{figure}

\begin{figure}
    \centering
    \includegraphics[width=0.7\textwidth]{Diag_UC_Admin.png}
    \caption{Diagramme de Cas D'utilisation Serveur}\label{uc_serv}
\end{figure}

\begin{figure}
    \centering
    \includegraphics[width=0.7\textwidth]{Diag_Class_Admin.png}
    \caption{Diagramme de Classes Serveur}\label{seq_serv}
\end{figure}

\begin{figure}
    \centering
    \includegraphics[width=1\textwidth]{Diag_Obj_Admin.png}
    \caption{Diagramme d'Objet Serveur}\label{obj_serv}
\end{figure}

\begin{figure}
    \centering
    \includegraphics[width=0.7\textwidth]{diag_seq_admin_consulter.png}
    \caption{Diagramme de séquence Serveur UC Consulter Données Utilisateurs }\label{seq1_serv}
\end{figure}

\begin{figure}
    \centering
    \includegraphics[width=0.7\textwidth]{diag_seq_admin_modif.png}
    \caption{Diagramme de séquence Serveur UC Modifier l'Environnement }\label{seq2_serv}
\end{figure}


\begin{figure}
    \centering
    \includegraphics[width=0.5\textwidth]{Diag_CS_User.png}
    \caption{Diagramme de Contexte Statique Client }\label{cs_client}
\end{figure}

\begin{figure}
    \centering
    \includegraphics[width=0.7\textwidth]{diag_uc_client.png}
    \caption{Diagramme de Cas d'Utilisations Client }\label{uc_client}
\end{figure}


\begin{figure}
    \centering
    \includegraphics[width=1\textwidth]{Diag_Classe_User_v3.png}
    \caption{Diagramme de Classes Client }\label{class_client}
\end{figure}

\begin{figure}
    \centering
    \includegraphics[width=1.2\textwidth]{Diag_obj_Client_v3.png}
    \caption{Diagramme d'Objets Client }\label{obj_client}
\end{figure}

\begin{figure}
    \centering
    \includegraphics[width=0.4\textwidth]{diag_seq_piloter_hélico.png}
    \caption{Diagramme de Séquence Client UC "Piloter hélicoptère" }\label{seq1_client}
\end{figure}

\begin{figure}
    \centering
    \includegraphics[width=0.4\textwidth]{diag_seq_piloter_rover.png}
    \caption{Diagramme de Séquence Client UC "Piloter rover" }\label{seq2_client}
\end{figure}

\end{document}